% xetex derivations && open derivations.pdf

\input xpmath
\input a4

\centerline{\bf Simulating the false vacuum with cold bosons}
\vskip 7mm
\centerline{Rodney E. S. Polkinghorne, Swinburne University of Technology}
\vskip 10mm

The Gross-Pitaevskii equations from Sascha's notes are 
$$i\hbar ∂₀ψ=\biggl(-{\hbar²\over 2m}∂_{11}-μ+g'|ψ|²+g_c|\bar ψ|²\bigg)ψ-ν\bar ψ,$$ where $ψ=(ψ⁰,ψ¹)$, and $\bar ψ=(ψ¹,ψ⁰)$, and absolute values are taken elementally.  Of the parameters, only $g'$ is a vector: $μ$, $g_c$ and $ν$ are scalars.  Only $ν$ depends on time, as $ν=ν₀+αω\cos(ωt)$.  For simplicity, we'll consider the case where $g⁰=g¹=g$, and $g_c=0$.

\beginsection{Literature review}

At the start of our journey of discovery, the terrain has been trampled by field theorists, several armies of mathematicians, Landau, Lifshitz, and assorted sightseers.  These hordes have left many tracks: some offer us essential shortcuts, many end in chaos.  Hence the map below.  Top left is most realistic, bottom right is most approximate.  Left is quantum, right is classical or mean-field.  Down is other kinds of approximation, labeled in bold and explained below.

\vskip 10mm
\centerline{\XeTeXpdffile "approx-1.pdf"}
\vskip 10mm

Experiment is, of course, the most realistic.  We expect the predictions of quantum mechanics to agree very well with the results of these tunneling experiments, so I've put it in the “quantum” column.  The most realistic theory is the atomic field theory that we intend to simulate.  Approximation~1 comprises the differences between the quantum field theory and the experiments.  The major ones are that we ignore losses, ignore the trap potential, and impose periodic boundary conditions.  On the right is the mean-field version of this theory, the Gross-Pitaevskii equations.

Approximation~2 is that the amplitudes of the two components of the field are almost homogeneous, and the fluctuations are slaved to the difference between their phases.  This gives the sine-Gordon equation for the mean field, and the corresponding field theory in the quantum case.

Approximation~3 makes the extent of the field small enough that it can be treated as a point.  This is where the system reduces to a single pendulum.

At level 3, there are several different theories describing each point.  I've divided these vertically by driving, and horizontally by linearisation.  The lower cases are the static ones, where the driving amplitude goes to zero.  This gives a pendulum, which reduces to an oscillator when linearised.  The upper cases are the adiabatic limit, where the driving frequency is much faster than any other timescale.  This is where the Kapitza pendulum lies.  We can model a dynamically stabilised false vacuum by solving the Kapitza potential quantum mechanically; it would be nice to check that this is the adiabatic limit of the driven quantum pendulum.  The adiabatic limit of the driven linear oscillator is the same as the static one, so I've left those corners blank.

So far as I know, the existing theories at level 2 are all in the static limit.  But thousands of papers have been written about sine-Gordon fields, and it seems very likely to me that someone would have studied a $\cos φ + ε\cos 2φ$ potential.  Finding that work will be hard.

Italics indicate cases that can be solved exactly, in the static or adiabatic limits.  It does not follow that driven systems between these limits can be solved exactly: there lies chaos!  Dashed lines show where chaos is expected.

For computational purposes, it's important to know what parameters govern the behaviour of each system.  Beneath each static, nonlinear system, I've listed the parameters that can't be removed by scaling.  In general, the linearised system has one less parameter.  The adiabatic system has one extra one, a product of modulation depth and frequency.  In driving at finite frequency, these parameters affect the dynamics independently.  In principle, the extent of the fields is an extra parameter.  I'm assuming they are large enough that it doesn't matter.

The classical pendulum has no meaningful parameters: there is only one set of dyanamics, and only the scale of the phase space changes.  When we quantise it, we get one parameter, which can be understood as follows.  There is a special trajectory, the one that starts and ends with the pendulum at rest pointing upwards.  This trajectory describes an orbit in phase space, whose area is an action, $L₀$.  The ratio of this action to Planck's constant matters.

The sine-Gordon equation is also parameterless.  Space and time can be scaled so that the speed of sound, and the length of a soliton, are both 1.  The quantum case is similar to the quantum pendulum.  Instead of a trajectory where the pendulum does a loop, we have a trajectory where a soliton-antisoliton pair annihilates, then recreates itself with opposite twist.  I don't know precisely what action characterises that process, but there must be one.  This is the parameter of the sine-Gordon field theory.  (According to Peter, the use of two parameters in this theory is for purely historical reasons.)

The Gross-Pitaevskii equations have a parameter, unlike the classical systems at levels 2 and 3 of approximation.  They have to: it's the parameter that determines how well the sine-Gordon approximation holds!  This parameter is something like particle number times repulsion strength divided by tunneling rate; Sascha claims the S-G limit is when the repulsion is much stronger than the tunneling.  There might be other parameters, such as the different repulsions for different species.

The atomic field theory has the $Ng/ν$ parameter, and also solition action parameter of the sine-Gordon field theory.  I don't know how to write this action in terms of $N$, $g$ and $ν$, because I don't fully understand the derivation of the sine-Gordon equation from the G-P equations.

\vfil\break

We're thinking about two pencils of condensed Bose gas, between which particles can tunnel.  People have been interested in this largely because, in the limit of nearly-uniform density, it should reduce to a sine-Gordon field, one of the sexiest topics around.

In the classical limit, a sine-Gordon field obeys the sine-Gordon equation, $$(∂_{00}-∂_{11})Φ=\sin φ.$$  This represents a very simple mechanical model: it is the continuous limit of a line of pendula connected by a torsional string.  The equation was of great interest when solitions were in fashion; it was solved exactly by Ablowitz et al.\ in 1973.

The corresponding quantum field theory is also of great interest, because it is exactly soluble, and relativistic.  This is hard-core relativistic field theory, which I don't know very much about.  The critical paper is Coleman in 1975, who showed it is equivalent to the Thirring model.  These models were both solved around the early 80's, using the Bete ansatz.  The solutions are taught in textooks, which Swinburne of course does not have.

Note that the quantum field theory has a meaningful parameter.  It's roughly the angular momentum kick that will get a downward-pointing pendulum to point upwards, in units of $\hbar$.  The classical sine-Gordon equation is parameterless.

We want to stabilise the upward equilibrium point by Kaptiza modulation.  This is more thoroughly trampled ground—it's a running exercise in Landau \& Lifshitz, and the chaos theorists discovered it around 1980.

The effect of Kaptiza modulation, in the adiabatic limit, is to add a term proportional to $\sin 2φ$ to the sine-Gordon equation.  One would think that, of the ten-thousand or so field theorists who've written about quantum sine-Gordon, at least one might have considered this.  Perhaps I should look at the intersection of citations from Coleman's sine-Gordon paper, and his false vacuum paper.


\beginsection{Analytical mechanics}

We aim to test various hypotheses about the dynamics of this system.  These hypotheses relate to a simple mechanical model, sine-Gordon, which is equivalent to the tunnelling atoms in some limit.  The classical version is very easy to visualise, and was solved by Ablowitz et.\ al in 1973.  The quantum version is exactly soluble, so we can rigorously test the truncation error in truncated Wigner, in at least one quantum limit.  The model is expressed in terms of generalised co-ordinates, so it is convenient to refer the different representations to a common Lagrangian.

The Euler-Lagrange equations for a field are derivied in Section~13.2 of Goldstein.  According to Kaurov and Kuklov, the mean-field Lagrangian for the two-component BEC is 
$${\cal L}(t, x, ψ⁰, ψ¹, {\dot ψ}⁰, {\dot ψ}¹, {ψ⁰}', {ψ¹}')={\cal L}₀+{\cal L}₁+{\cal L}₂,$$
$${\cal L}₀+{\cal L}₁ =∑_{ψ∈ψ₀,ψ₁}{i\hbar\over 2}(ψ*{\dot ψ}-ψ{\dot ψ}*) - {\hbar²\over 2m}|∂₁ψ|²+μ|ψ|²-{g\over 2}|ψ|⁴,$$
$${\cal L}₂=γ(ψ₀*ψ₁+ψ₀ψ₁*).$$
Here, for a candidate solution $ψ(t,x)$, $${\dot ψ}(t,x) = ∂₀ψ(t,x) \quad {\rm and} \quad ψ'(t,x) = ∂₁ψ(t,x).$$  For fields, the roles of position and time a symmetric.  The Euler-Lagrange equations are then $$ foo $$

\beginsection{Preliminaries}

Transforming to dimensionless coordinates, $$ψ(t,x)=q\bigl(ωt, \sqrt{mμ}x/\hbar\bigr)=q(τ,χ),$$ gives 
$$i\hbar ω∂₀q=\bigl(-½μ∂_{11}-μ+g'|q|²+g_c|\bar q|²\bigr)q-ν\bar q,$$ or
$$∂₀q=is\bigl(½∂_{11}+1-g|q|²-f|\bar q|²\bigr)q+i(b+a\cos\tau)\bar q,$$ where $$s=μ/\hbar ω\qquad g=g'/μ\qquad f=g_c/μ\qquad b=ν₀/\hbar ω\qquad a=α/\hbar.$$  As stated, $q$ has dimensions ${\rm m}^{-½}$; fixing that will change the values and dimensions of $g$ and $f$, but not the form of the equation.

The inital amplitude of $q$ is set consistently with the chemical potential, so that a flat field is a steady state, with $q⁰=q$ and $q¹=cqe^{iφ}$:
$$\bigl(gq²+fc²q²-1\bigr)=(b/s)ce^{iφ},$$
$$\bigl(gc²q²+fq²-1\bigr)c=(b/s)e^{-iφ}.$$
The only stationary solutions occur with $φ=0,π$ and $c=1$, whence
$$q=\sqrt{1±b/s\over g+f}.$$
The plus sign applies to $φ=0$, the minus sign to $φ=π$.

\beginsection{Scales}

There are three quantities to scale in this problem: length, time, and the ampltiude of $ψ$.

One set of length and time scales come from the sine-Gordon equation.  This involves a speed of sound, and a natural frequency for the coupled pendula; physically, this is a Rabi frequency for the tunneling.  These scales make the most sense for computing dynamics and stability: if the system is unstable, it is likely to escape on a time around the pendulum period.

The other length scales come from statics.  But these should be captured by the sine-Gordon equation.  I expect the healing length is the speed of sound divided by the Rabi frequency, to a factor of $π$ or something.

The field amplitude is, in principle, determined by the chemical potential $μ$.  For the Gross-Pitaevskii equations to be truly stationary, $μ$, the tunneling rate $ν$, and the repulsion $g|ψ|²$ must satisfy some constraint.  However, changing $μ$ multiplies $ψ$ by a global phase $e^{iμt}$; in simulations, $μ$ will be adjusted so this global phase rotates as slowly as possible.

We will consider fields that are largely homogeneous.  The amplitude of the homogeneous part can then be scaled so that $|ψ⁰|²+|ψ¹|²=1$; the only change in the equations for $g$ to be multiplied by the particle density.  This combines $N$ and $g$, eliminating one parameter.  Length could be scaled consistently, so that $|ψ|²\,dx$ was still the number of particles in the interval $dx$.  However, it would make more sense to use an unusual normalisation, to give the sine-Gordon equation the above form.

\beginsection{Dynamic stabilisation}

The pendulum with vertically vibrating support was solved by P.~L.~Kapitza in 1951.  The solution is most accessible as a running exercise in Landau \& Lifshitz; the means to solve it are developed in \S 30, and the solution occurs as Problem~1 on Page~95.

\beginsection{Questions}

In the usual derivation of the Euler-Lagrange equations for a field, it is assume that the variation of the field is zero at its edges.  How do periodic boundary conditions work?

\bye